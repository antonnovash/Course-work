\chapter{ОСНОВНЫЕ ПОНЯТИЯ И ОБЗОР ЛИТЕРАТУРЫ}\label{chap1}

Первая глава всегда посвящена обзору литературы. В начале каждой главы необходимо написать небольшую аннотацию о содержании главы (так называемая \textit{врезка}). Например:

В настоящей глава формулируются основные понятия, используемые в диссертации. Приводится классификация (согласно работе \cite{GabasovKirillovaPU}) принципов управления, используемых в современной теории управления. Объясняется принцип управления в режиме реального времени в применении к реализации оптимальных обратных связей в задачах оптимального управления с конечным горизонтом планирования \cite{GabasovDmitrukKirillova15a} и в применении к задачам стабилизации нелинейных объектов согласно теории  управления по прогнозирующей модели \cite{Mayne}.


%%%%%%%%%%%%%%%%%%%%%%%%%%%%%%%%%%%%%%%%%%%%%%%%%%%%%%%%%%%%%%%%%%%%%%%%%%%%%%%%
\section{Оптимальное управление в  реальном времени}\label{1sec:optimal-control}
%%%%%%%%%%%%%%%%%%%%%%%%%%%%%%%%%%%%%%%%%%%%%%%%%%%%%%%%%%%%%%%%%%%%%%%%%%%%%%%%


Рассмотрим задачу оптимального управления
\begin{equation} \label{1problem}
    c^Tx(t_f)\to \min,
    \end{equation}
$$
    \dot{x}=A(t)x+B(t)u,\ x(t_0)=x_0^*,
    $$
$$
    x(t_f) \in X_f,\quad  u(t)\in U, \quad  t\in T,
    $$
где  $X_f=\{x\in \mathbb{R}^n: g_*\leq Hx \leq g^*\}$ --- терминальное
множество, $H\in \mathbb{R}^{m\times n}$, $g_*,$ $g^* \in \mathbb{R}^m$;
$U=\{u\in \mathbb{R}^r: u_*\le u\le u^*\}$ --- множество доступных значений
управляющего воздействия.

\begin{definition}  Управляющее воздействие $u(t)\in U$, $t\in T$, называется программой, если соответствующая ему
траектория $x(t)$, $t\in T$, математической модели (\ref{1problem}) удовлетворяет
условию $x(t_f)\in X_f$.
\end{definition}

\begin{definition}  Программа $u^0(t)$, $t\in T$,
называется оптимальной (программным решением задачи (\ref{1problem})), если на соответствующей ей (оптимальной) траектории $x^0(t)$, $t\in T$, выполняется равенство
$$
    c^Tx^0(t_f) = \min_u c^Tx(t_f),
     $$
где минимум ищется среди всех программ.
\end{definition}

........


%%%%%%%%%%%%%%%%%%%%%%%%%%%%%%%%%%%%%%%%%%%%%%%%%%%%%%%%%%%%%%%%%%%%%%%%%%%%%%%%
\section{Теория управления по прогнозирующей модели}\label{1sec:MPC}
%%%%%%%%%%%%%%%%%%%%%%%%%%%%%%%%%%%%%%%%%%%%%%%%%%%%%%%%%%%%%%%%%%%%%%%%%%%%%%%%

В западной литературе (см. \cite{Keerthi, Mayne2000, Mayne}) управление в реальном времени представлено теорией управления по прогнозирующей модели --- Model Predictive Control (MPC), также называемая Receding Horizon Control (RHC). Основными приложениями теории являются задачи стабилизации динамических систем. Современная теория нелинейного MPC предлагает основанные на решении задач оптимального управления методы построения обратных связей для нелинейных объектов.

Главная идея MPC --- использование математической модели управляемого процесса в пространстве состояний для предсказания и оптимизации будущего поведения системы. Поясним на примере модели нелинейного процесса управления
\begin{equation}\label{sys2}
    \dot x = f(x,u)
    \end{equation}
где $x=x(t)\in \mathbb{R}^n$ --- состояние модели  в момент времени $t$; $u=u(t)\in
\mathbb{R}^r$ --- значение управляющего воздействия; $f:\mathbb{R}^n \to \mathbb{R}^r$ --- заданная функция, обеспечивающая существование и единственность решения (\ref{sys2}) при любом допустимом управляющем воздействии.


........

\bigskip

Каждая глава завершается краткими выводами. Разумный способ написания выводов --- переписать (это значит использовать те же мысли, но не копировать фразы!) в утвердительной форме (рассмотрено, получено и т.д.) то, что написано во врезке. 