\chapter{ОПТИМАЛЬНОЕ ЦЕНТРАЛИЗОВАННОЕ УПРАВЛЕНИЕ В ЭКОНОМИЧЕСКОЙ ЗАДАЧЕ}\label{chap2}
В данной главе описывается сведение задачи оптимального управления к задаче линейного программирования. Формулируется и решается линейная задача оптимального управления.Рассматривается алгоритм централизованного управления.
\section{Построение программных решений}\label{1sec:program-solution}
Кратко остановимся на методе решения задачи (\ref{5problem}). В классе дискретных управлений задача (\ref{5problem}) сводится к задаче линейного программирования. Для этого запишем формулу Коши для линейной системы. 

\begin{equation} \label{2.1}
x(t^*) = F(t^*,\tau)x^*(\tau) + \int_{\tau}^{t^*} F(t^*,t)B(t)u(t)\dt ,
\end{equation}
где $F(t^*,t)$ — фундаментальная матрица, т.е. решения уравнения
$$\dot F(t^*,t) = - F(t^*,t)A(t),\quad F(t^*,t^*) = E.$$
Инеграл в правой часть равенства (\ref{2.1}) запишем следующим образом:  
\begin{equation} \label{2.2}
x(t^*) = F(t^*,\tau)x^*(\tau) +  \sum_{s=\tau}^{t^*-h}\int_{s}^{s+h} F(t^*,t)B(t)\,dt   u(s).
\end{equation}
Используя (\ref{2.2}) запишем левую часть терминального ограничения:

\begin{equation} \label{2.3}
Hx(t^*) = HF(t^*,\tau)x^*(\tau) +  \sum_{s=\tau}^{t^*-h}\int_{s}^{s+h} HF(t^*,t)B(t)\,dt   u(s).
\end{equation}

Равенство  (\ref{2.3}) перепишем в виде:

$$Hx(t^*) = \Phi(\tau)x^*(\tau) +  \sum_{s=\tau}^{t^*-h}D(s)u(s)$$.

где введены следующие обозначения:

$$\Phi(t) = HF(t^*,t) \quad \Phi(t^*) = H, \quad \dot{\Phi} = -\Phi A(t);$$
$$D(s) =\int_{s}^{s+h} \Phi(t)B(t)\,dt,\quad s\in T_h .$$

Таким образом,исходная задача (\ref{5problem}) эквивалентна следующей задаче линейного программирования:

\begin{equation} \label{9problem}
\sum_{t\in T_h(\tau)}^{}c'(t)u(t) \to \min,  
\end{equation}
$$g_*(\tau) \leq \sum_{s\in T_h(\tau)}^{}D(s)u(s) \leq g^*(\tau),$$
$$ u_*\leq u(t) \leq u^*,\quad t \in T_h(\tau)$$.

где
$$c'(t) = \int_{s}^{s+h} F(t^*,t)B(t)\,dt,\quad s\in T_h$,$

$$g_*(\tau) = g_* - \Phi(\tau)x^*(\tau),\quad g^* - \Phi(\tau)x^*(\tau).$$


%\section{Алгоритм централизованного управления}\label{1sec:algoritm}